
% this file is called up by thesis.tex
% content in this file will be fed into the main document

%: ----------------------- introduction file header -----------------------
\chapter{Large-scale Comparisons of Topic Distributions}\label{ch:comparisons}

\graphicspath{{comparisons/figures/}}

% -------------------------------------------------------------
% -- Comparisons
% -------------------------------------------------------------

As we showed in Section \ref{sec:topic-clustering}, topic groupings based on cumulative ranking are useful mechanisms for simplifying representations based on topic distributions. Relevant topics are those whose accumulated weight exceeds a threshold, starting with those who have more, and have shown a promising performance to cluster documents ( Section \ref{sec:clustering-results}). This suggests that \textit{similar documents share the most relevant topics}. However, this definition has two main limitations: it depends on the manual tuning of a parameter, the threshold; and it does not measure degrees of similarity, it only establishes whether or not two documents are similar. As shown in Chapter {ch:hypothesis}, we hypothesize that is possible to find relevant documents with similar topic distributions without calculating all pairwise comparisons and without discarding the notion of topics from their representation. 

\section{Document Similarity}

\section{Hashing Topic Distributions}

\section{Summary}