
% this file is called up by thesis.tex
% content in this file will be fed into the main document

%: ----------------------- introduction file header -----------------------
\chapter{Conclusions and Future Work}\label{ch:conclusion}

\graphicspath{{conclusions/figures/}}

This thesis addresses different challenges to facilitate the exploration of large multilingual document corpora through the use of probabilistic topic models. Four main research problems arise from them, as we discussed in Section \ref{sec:research-challenges}. The first one is the \textbf{efficiency} to create and infer probabilistic topics. There are some technical barriers that may limit a wider use of topic models in high-volume scenarios. The second challenge is the \textbf{explainability} of topic-based relations among documents. It is difficult to understand how two documents are related from the numerical distance of their vector representations. The third challenge is the \textbf{complexity} of comparisons among topic distributions at large-scale. Brute-force approaches are not feasible for big documentary corpora. And finally, the fourth challenge is \textbf{multilinguality} when comparing topic distributions from documents written in different languages. In order to be able to work with multilingual corpora at any scale, it is necessary to avoid the need for translators or annotations between languages that may limit their feasibility.

As shown in this thesis, the main hypothesis has proven to be true, so \textit{large multilingual document collections can be automatically analyzed to discover thematic representations that enable an exploration through related texts} (H1). Based on the evaluation of our results, probabilistic topic models enable an unsupervised exploration of corpora containing a huge number of texts written in different languages. We review below the research problems mentioned above through the hypotheses evaluated in this work and the contributions we offer to address them. The rest of the section discusses the impact, current limitations and future lines of our work. 


\section{Contributions}


The main contribution of this thesis is the \textbf{librAIry framework}, a system that processes and analyzes huge collections of textual resources creating and using probabilistic topic models. This framework encompasses several contributions that are aimed at addressing the four aforementioned research problems. Following the methodology described in Section \ref{sec:research-methodology}, we have evaluated our hypotheses using \textit{librAIry} for large multilingual document corpora. 

\subsection{Efficient Creation and Use of Probabilistic Topic Models}

As discussed in Chapter \ref{ch:scalability}, previous works dealing with probabilistic topic models are mostly focused on improving the learning process and ignore other important features related to its development in potentially heterogeneous scenarios. The creation of a topic model (Fig. \ref{fig:life-cycle}) covers a first stage of \textit{document preparation}, where texts are pre-processed to create bags-of-words. The next stage (\textit{training} stage),  builds a model based on patterns among word distributions. The model is then packaged for distribution in the (\textit{publication} stage). And finally the model can be used and reused in the (\textit{exploitation} stage). Our objective in this thesis is to facilitate the creation of reusable probabilistic topic models by minimizing their technical dependencies for use in both large- and small-scale contexts. But we have identified some research challenges that make it difficult to achieve that goal:  \textit{reuse of topic models is limited by incompatibility problems} (RCInterface1), and \textit{there is no unified or standardized format for distributing topic models} (RCInterface2). Along these research challenges we have formulated one hypothesis and proposed a series of contributions that we discuss below.

\textbf{H1.1 \textit{Documents can be efficiently annotated on a large scale by distributing across different computation nodes both natural language processing tasks and topic models}}. This hypothesis is motivated by the limitations of existing works identified and discussed in Section \ref{sec:topic-reuse} that can be summarized as follows:

\textbf{Limitation 1}: Summaries or sections are usually considered to describe documents using probabilistic topics due to technical difficulties in processing full-texts in large collections. However, mining full-text articles gives consistently better results \citep{Westergaard2017}.

\textbf{Contribution 1}: We have created librAIry, a topic modeling framework introduced in Section \ref{sec:topic-model-framework} to overcome this limitation that has been validated in real world scenarios with large document collections. It proposes an event-driven architecture for text processing and topic inference that adapts its workload to the size of the corpus. Initially, data is organized into \textit{snippets} (to describe pieces of texts), \textit{documents} (to represent full texts), and \textit{domains} (to group documents), but it has evolved to a more reduced representation of \textit{texts} and \textit{collections}. librAIry has been tested as part of the Corpus Viewer platform \citep{Samy2019}, where it was used to analyze the current situation and trends of information and communication technologies (ICT) through the study of patent collections and grants for R\&D projects (see Section \ref{sec:corpus-viewer}). We have also used this implementation to support complex calculations on data sets from different domains. For example, to relate patients according to the medicines they receive \citep{Badenes-Olmedo2019c} (see Section 	\ref{sec:polypharmacy}), or to relate medicines or diseases from the experiments where they are used (see Section \ref{sec:drugs4covid}). 


\textbf{Limitation 2}: APIs based on topic models define their own distribution formats limiting the interoperability of the models \citep{Lisena:NLPOSS2020}. To the best of our knowledge, the efforts made do not propose an unified model to exchange topic models, understood as an already accepted standards-based format.

\textbf{Contribution 2}: Our second contribution consists of a method to make topic models openly accessible as web resources. In Section \ref{sec:reusable-topic-modeling} we propose a Web service template based on REST principles to homogenize the format of topic models and facilitate their usage. This allows easy accessibility and reuse for both humans and machines aiming to consume topic model related information. As described in Section \ref{sec:topic-model-publication} three tasks guide the creation of a topic model as web service: \textit{reproducibility}, \textit{exploration} and \textit{inference}.The list of available methods for using a topic model is provided in Table \ref{table:operations}. Finally, an online repository has been also proposed in Section \ref{sec:topic-model-exploitation} to promote the reuse of existing models as virtual services that have meta-information about their training process and multiple versions. 

\subsection{Explainability of Topic-based Relations among Documents}

As discussed in Chapter \ref{ch:explainability}, state-of-the-art metrics for comparing topic distributions are difficult to understand. Since topic distributions are density functions, the distance is calculated by aggregating the intersections of each dimension of the vector, i.e., of each topic. However, as seen in Section \ref{sec:topic-explainability}, high dimensional models create more specific topics than models with fewer dimensions, and this topic specificity influences the way in which topic distributions are related, and consequently how documents can be related. The same pair of documents may vary their distance from each other when using topic models with different dimensions to represent them (Figure \ref{fig:topic_distances}). Our objective in this thesis is to describe texts based only on the most representative topics and compare documents taking into account these representations. But we have identified some research challenges that make it difficult to achieve that goal: \textit{there is no common criteria for identifying the most representative topics in a document} (RCExplainable1), \textit{it is difficult to understand the distance between topic distributions} (RCExplainable2) and \textit{there is no common criterion for determining wheter documents are related} (RCExplainable3). Along these research challenges we have formulated one hypothesis and proposed a series of contributions that we discuss below.

\textbf{H1.2 \textit{It is possible to semantically relate documents by comparing their most relevant topics}}. This hypothesis is motivated by the limitations of existing works identified and discussed in Section \ref{sec:topic-explainability} that can be summarized as follows:

\textbf{Limitation 3}: We analyzed the behavior of topic distributions when using topic models with different dimensions, i.e. different number of topics. After our evaluations we saw that the most relevant topics cannot be identified by fixed thresholds, since as the dimensions of the model vary the relative weights also vary. Nor can they be identified using clustering techniques based on centroids, since the number of groups with homogeneous weights is unknown a priori.

\textbf{Contribution 3}: We proposed a method to identify the most relevant topics using clustering techniques based on densities at the topic level. Topics are grouped into three levels of relevance based on their weights. The less relevant topics are discarded to represent the document. The most representative topics belong to one of these three levels. This method has proven to be robust against changes in the number of topics, i.e. model dimensionality.

\textbf{Limitation 4}: State-of-the art distance metrics among topic distributions are based on density functions, not on sets of topics according to their relevance. The representation of documents in these cases is not based on weighted vectors, but on sets and levels.

\textbf{Contribution 4}: A new distance metric that relates similar texts is proposed based on the most relevant topics. The distance between two texts is proportional to the number of topics they share at the same relevance level. Its performance is evaluated in unsupervised classification tasks and shows (Tables \ref{tab:precisionHe}, \ref{tab:precisionJS}, \ref{tab:recallHe} and \ref{tab:recallJS}) promising results with high precision and recall values. The corpus used was the JRC-Acquis with annotations in EUROVOC categories.

\subsection{Data Structures and Algorithms for Large-Scale Comparisons of Documents}

Brute-force techniques cannot be applied to compare all items in a huge corpus. Our two main contributions to this research challenge are a data structure to describe documents from their most relevant topics, that partially addresses the RCComparison1 (\textit{there are no mechanisms that efficiently partition the topic-based search space without compromising the ability for thematic exploration}), and an efficient algorithm to make comparisons on a large scale based on topic relevance, that addresses RCComparison2 (\textit{there are no similarity metrics that compare partial distributions of topics}).

\subsubsection{Topic Hierarchies Representation}

We created a new data structure to represent topic distributions as topic hierarchies that uses the relevance of each topic to define hierarchy levels. This way of encoding documents has also helped to understand why two documents are similar, based on the intersection of topics at hierarchies of relevance.

The approach can accommodate additional query restrictions when searching for related documents (e.g. documents that mainly deal with one theme, although they also deal with another) and has proven to obtain high-precision results. 

\subsubsection{Comparisons based on Topic Hierarchies}

With large amounts of items in a collection, discovering the entire set of nearest neighbors to a given document is infeasible. Due to the low storage cost and fast retrieval speed, hashing is one the popular solutions for approximate nearest neighbours. However, existing hashing methods for probability distributions only focus on the efficiency of searches from a given document, without handling complex queries or offering hints about why one document is considered more similar than another.

We developed a method to compare and organize huge document collections based on similar topic hierarchies. The hierarchy levels are compared and the distance between texts depends on the degree of intersection between pair of representations.

In addition, the technique to represent and compare documents has been implemented in our librAIry framework.

\subsection{Multilingual Corpora}

Document in different languages must be described by multilingual topics to be thematically related without having to translate their texts. However, parallel or comparable corpora, or multilingual dictionaries are required to abstract topics into a language-independent space. Our main contribution to this research challenge is a method that automatically creates a topic-based space shared among all languages from the language-dependent models independently created. It address the RCCrossLingual1 (\textit{there are no approaches to abstract probabilistic topics in language-independent spaces without translating text or aligning documents}) 

\subsubsection{Cross-lingual Topics}

We proposed a unique space to describe documents based on topic models created from different languages. Topics are created independently for each language, and are projected on concepts instead of words. On concept-based representations, documents in different languages coexist together and can be related.

Representations are analyzed in classification and information retrieval tasks on multilingual document collections. As expected, the performance in terms of accuracy is not as good as that of the approach based on prior knowledge (i.e. topics previously aligned by documents annotated with categories). However, in terms of coverage, the performance of the unsupervised approach is much greater than that offered by the semi-supervised approach, to the point of offering better overall performance (i.e f1) in classification tasks. 

In addition, the algorithm proved to perform close to the semi-supervised algorithm in information retrieval task, which makes us think that the process of topic annotation by set of synonyms (i.e. concepts) can be improved to filter those elements that are not sufficiently representative.


\section{Impact}

The contributions of this thesis have already started to be used in recent work by other researchers. The librAIry framework was used, among others, to create, publish, distribute and reuse probabilistic topic models for studing the Language Technologies (TL) sector in Spain \citep{Samy2019}. The analysis took into account  the structured and unstructured data from the ACL Anthology repository in order to portray the current panorama in terms of underlying topics and their evolution in recent years in comparison with the international community. The framework has also been used to facilitate the classification of public offering using big data techniques \citep{Olga2019} and to analyze texts in reading support systems \citep{Teresa2020}.

Our methods to categorize documents through topic hierarchies and to make efficient large-scale comparisons  have been adapted to other domains. In \citep{Badenes-Olmedo2019c} they were adapted to compare patients from representations based on the medications they were using. The approach is very similar since drugs, like topics, can be distributed in different proportions according to the patient. To take advantage of the hierarchical representation of the topics, a more complex approach based on drug-drug interactions was performed (see Section \ref{sec:polypharmacy}).

Other contributions may impact different research areas, as we describe below:
\begin{itemize}
\item \textbf{Corpus Exploration}: By knowing how topics are present in a corpus in terms of relevance, it is possible to derive a thematic network where the nodes represent the different topics and the edges are the documents containing them on the same level of relevance. Thus, the higher the number of common documents the stronger the connection would be between two topics. This kind of network would help users to see how different topics are connected with each other in a corpus, facilitating the exploration between topics.
\item \textbf{Topic Discovery}: Commonly occurring topics, when supported in many documents, may become valuable topic candidates themselves. Documentary management system may offer these potentially new topics to users to include in their corpus description.
\item \textbf{Language Abstraction}: Approaches like \citep{hao-paul-2018-learning} create multilingual topics from incomparable corpora. These approaches can benefit from our hierarchical representations to create language-independent spaces where documents are described by their most relevant topics.
\item \textbf{Corpus Visualization}: As stated in Section \ref{ch:comparisons}, one of the reasons why groupings are created is for summarization and organization purposes. Hence, commonly occurring topics by hierarchies can be used to simplify the complexity of a corpus by collapsing their documents under a single group. If a corpus has overlapping clusters it would be possible to create different views according to user preferences, simplifying the overall complexity shown to the final user.
\item \textbf{Corpus Compression}: Similarly, if several documents share the same most relevant topics, it would be possible to store the common relevant topics instead of every unique topic distribution, for efficiency. This would be particularly useful when dealing with similar or identical texts.
\item \textbf{Topic Suggestions}: Commonly relevant topics may be used to suggest how a user may complete the writing of a document. By comparing the current topic hierarchies with the commonly relevant topics, it would be possible to recommend the next topic o topics to be considered in the document.
\item \textbf{Document Ranking}: Once the topics are described by hierarchies of relevance, it would be possible to order the documents in a corpus by different criteria and create rankings. Possible examples of rankings are basic searches based on one or several highly relevant topics, or more complex searches that combine different topics and different degrees of relevance.  
\end{itemize}

%\section{Limitations}
%...

\section{Future Work}

Our work may be expanded in several ways. In this section we discuss possible future lines of work related to our research challenges.

\begin{itemize}
\item \textbf{Extend the Reuse of Topic Models}: There are several ways in which our topic services may be improved to be more useful. First, it would be necessary to study the different tasks where a topic model can be used in order to allow users to customize their inferences. As we have seen in our analysis, classification and information retrieval tasks require topic distributions as vectors of weights and also as hierarchies of relevance, as well as having the weights of the words for each topic. Other metadata like the corpus used to train it, accessible and downloadable, or the libraries used to create the model may be relevant for facilitate reproducibility and improvement of existing models. In this line, the next step is to promote the standardization of these models from the W3C or IETF communities. The objective is to agree on a distribution format for probabilistic topic models that facilitates their generation, publication and use.

Second, it should be possible to rank topic models once they have been created and published. Rankings make a set of topic models easily comparable in order to recommend similar models covering a particular task (based on the same training set or similar categories). It would be an interesting line of work to determine if a recommendation based on topic models provides similar or better results than other approaches to solve the same problems.

Third, librAIry has currently implemented the L-LDA and LDA algorithms to create models with labeled or unlabeled topics, respectively. Thanks to its design, the addition of new techniques for learning different probabilistic topic models is straightforward. It only requires the creation of a web resource with support for all the methods that have been defined in the model-as-a-service web interface. In order to facilitate that process by adding models already trained with other techniques, we want to create templates (in the form of bash or python scripts) that automate the whole process.

Fourth, a set of topic models may be used for suggestions in corpus exploration, in combination with text mining approaches. Text mining approaches have been demonstrated to obtain good results to classify o retrieve texts from a corpus. When using text mining techniques with topic models as services, we suspect that good results may be obtained, as context information can be added using categories from these topics. 
\item \textbf{Combined Representation of Topic Models}: There are many of scenarios where big collections of documents, sometimes available in different languages, have to be browsed according to certain priorities/information needs, e.g. in the medical domain huge amounts of medical reports are daily created offering the possibility of extracting knowledge from them. How they can be explored and effectively consumed will depend on the medical specialty considered. 

Psychiatrists may be looking for collections of reports related by the patient\'s mood, while pharmacists may want to find patients with the same adverse reactions. In a first approach, only structured information are considered (e.g disease, patient, location, date, insurance). This limits the ability to relate reports with information not described by keywords or that cannot be matched because it is written in different languages (e.g. patient mood, response to medications, or behavior during treatment). 

In such situations it would be advantageous to be able to rely on language-agnostic and unsupervised annotation units derived from the text so the full content does not need to be translated, for instance by inferring the notion of topics discussed in the documents. In this research line, probabilistic topic models can open the door to a new set of possibilities to automatically learn cross-lingual concepts to browse multi-lingual medical report collections. Thanks to these techniques, patients who, having different illnesses, manifest similar behaviors from a particular point of view can thus be related (Figure \ref{fig:context-similarity}). 

The greater the presence or absence of concepts is leading the final similarity score, but it might be an oversimplification of what this approach is really doing as it can be based on topic hierarchies. Diagnoses could therefore be supported by collections of similar medical reports from multiple medical areas, in different languages, and in real-time. Thanks to approaches like this one, a pharmacist in Paris could find patients in India who have similar reactions without medication, described in their reports, to those detected when certain medications are prescribed. Similarly, a German psychiatrist is able to check how some pregnant women in England share similar symptoms to patients with anxiety problems after having heart surgery.
\item \textbf{Refine the Conceptual Abstractions of Topics}:In this thesis we have proposed a way of creating language-independent abstractions of probabilistic topics: representations based on concepts, instead of words, created by cognitive synonyms (i.e. synsets). Even though we have contributed towards the automatic alignment of multilingual topics, our work may be expanded, as we further describe below. 

As demonstrated in evaluations on document retrieval and classification tasks, the accuracy of our approach is reduced because concepts are more general than words to describe topics. The mapping between words and concepts to describe a topic should be reviewed to create representations that are general enough to relate topics from different languages, and precise enough to separate different topics. 

\end{itemize}


\begin{figure}[ht]
    \centering
    \includegraphics[width=0.7\linewidth]{context-similarity.png}
    \caption{Similarity among medical reports depends on the reference knowledge used to analyze them}
    \label{fig:context-similarity}
\end{figure}


