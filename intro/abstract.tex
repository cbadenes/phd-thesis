
% Thesis Abstract -----------------------------------------------------


\begin{abstractslong}  
%english
Searching for similar documents and exploring major themes covered across groups of documents are common activities when browsing collections of scientific papers. With the ongoing growth in number of digital articles in a wider set of languages and the expanding use of different languages, we need annotation methods that enable browsing multilingual corpora. This manual knowledge-intensive task can become less tedious and even lead to unexpected relevant findings if unsupervised algorithms are applied to help researchers. Most text mining algorithms represent documents in a common feature space that abstract them away from the specific sequence of words used in them. Probabilistic Topic Models reduce that feature space by annotating documents with thematic information. Over this low-dimensional latent space some algorithms have been proposed to perform document similarity search, even on collections of texts in multiple languages. However, thematic information gets hidden behind specific representations, preventing thematic exploration and limiting the explanatory capability of topics to justify content-based similarities. Furthermore, multilingual approaches require theme-aligned training data to create a language-independent space that limits the amount of scenarios where it can offer solutions and makes it difficult to scale up to situations where a huge collection of multilingual documents are required during the training phase. In this thesis we address the issue of automatically relating multilingual documents on a large scale without losing the knowledge offered by topics to explain the relationships. In order to do so, we propose a simple model for representing and relating documents by thematic annotations and a framework for creating and reusing probabilistic topic models, we analyze the representative capacity of topic models to describe the content of scientific publications, we create a novel representation based on topic hierarchies and a hashing algorithm based on approximate nearest-neighbor techniques that takes advantage of that hierarchical representations to relate documents, and we presents an unsupervised document similarity algorithm that does not require parallel or comparable corpora, or any other type of translation resource, to relate multilingual document on a large-scale. Extensive evaluations on multiple domains reveal promising results on classifying and sorting documents by similar content. Our methods not only performs efficient similarity searches, but also allows extending searching queries with thematic restrictions explaining the similarity score from the most relevant topics. 


\end{abstractslong}

\cleardoublepage
\begin{abstractslongSpanish}
%spanish
La búsqueda de documentos similares y la exploración de los principales temas tratados en los distintos grupos de documentos son actividades comunes cuando se examinan colecciones de documentos científicos. Con el continuo crecimiento del número de artículos digitales en un conjunto más amplio de idiomas y el uso cada vez mayor de diferentes idiomas, se necesitan métodos de anotación que permitan la navegación de corpus multilingües. Esta tarea manual de extracciómn de conocimiento puede hacerse menos tediosa e incluso conducir a nuevos e inesperados hallazgos si se aplican algoritmos no supervisados para ayudar a los investigadores. La mayoría de los algoritmos de minería de textos representan documentos en un espacio de características comunes que los abstraen de la secuencia específica de palabras utilizadas en ellos. Los modelos probabilísticos de tópicos reducen ese espacio de características anotando los documentos con información temática. En este espacio latente de reducidas dimensiones se han propuesto algunos algoritmos para realizar búsquedas de similitud de documentos, incluso en colecciones de textos en múltiples idiomas. Sin embargo, la información temática se oculta detrás de representaciones específicas, lo que impide la exploración temática y limita la capacidad explicativa de los temas para justificar las similitudes basadas en el contenido. Además, los enfoques multilingües requieren textos de entrenamiento que estén alineados temáticamente para crear espacios independientes del idioma, y esto limita la cantidad de escenarios en los que puede ofrecer soluciones y dificulta su utilización cuando se requiera una enorme colección de documentos multilingües. En esta tesis abordamos el problema de relacionar automáticamente documentos multilingües a gran escala sin perder el conocimiento que ofrecen los tópicos para explicar las relaciones. Para ello, proponemos un modelo sencillo donde representar y relacionar documentos mediante anotaciones temáticas y un marco de trabajo donde 
crear y reutilizar modelos probabilísticos de tópicos, analizamos la capacidad de representación de los modelos de tópicos para describir el contenido de publicaciones científicas, creamos una representación novedosa basada en jerarquías de tópicos y un algoritmo de hashing basado en técnicas de aproximación del vecino más cercano que aprovecha esas representaciones jerárquicas para relacionar documentos, y presentamos un algoritmo de similitud de documentos no supervisado que no requiere corpus paralelos o comparables, ni ningún otro tipo de recurso de traducción, para relacionar documentos multilingües a gran escala. Las evaluaciones exhaustivas en múltiples dominios revelan resultados prometedores al clasificar y ordenar documentos por su contenido. Nuestros métodos no sólo realizan búsquedas eficientes de similitud, sino que también permiten extender las consultas de búsqueda con restricciones temáticas que explican la semejanza a partir de los temas más relevantes. 

\end{abstractslongSpanish}
% ---------------------------------------------------------------------- 
