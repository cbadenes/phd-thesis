% this file is called up by thesis.tex
% content in this file will be fed into the main document

%: ----------------------- introduction file header -----------------------
\chapter{Acronyms } \label{ch:acronyms}

\graphicspath{{intro/figures/}}

% -------------------------------------------------------------
% ----------------------- Chapter intro ----------------------- 
% -------------------------------------------------------------


\begin{description}
\itemsep0em
	\item[AIES:] Advances In Engineering Software journal dataset
	\item[ANN:] Approximate Nearest Neighbour
	\item[API:] Application Programming Interface
	\item[BoW:] Bag-of-Words
	\item[CHHM:] Centroid-based Hierarchical Hashing
	\item[CRDC:] Cumulative Ranking on Dirichlet distribution- based Clustering 
	\item[DHHM:] Density-based Hierarchical Hashing
	\item[DRM:] Dirichlet Random Mixture dataset
	\item[HE:] Hellinger Distance
	\item[IR:] Information Retrieval
	\item[JS:] Jensen Shannon divergence
	\item[KL:] Kullback-Liebler divergence
	\item[LDA:] Latent Dirichlet Allocation
	\item[MAP:] Mean Average Precision
	\item[MuPTM:] Multilingual Probabilistic Topic Model
	\item[NLP:] Natural Language Processing
	\item[NM:] Neural Models
	\item[PTM:] Probabilistic Topic Model
	\item[RDC:] Ranking on Dirichlet distribution-based Clustering
	\item[TDC:] Trends on Dirichlet distribution-based Clustering
	\item[THHM:] Threshold-based Hierarchical Hashing
	\item[URI:] Uniform Resource Identifier
	\item[URL:] Uniform Resource Locator
	\item[VSM:] Vector Space Models
	\item[WUI:] Web User Interface
\end{description}









