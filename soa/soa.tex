
% this file is called up by thesis.tex
% content in this file will be fed into the main document

%: ----------------------- introduction file header -----------------------
\chapter{Related Work}\label{ch:soa}

\graphicspath{{soa/figures/}}

% -------------------------------------------------------------
% -- Related Work
% -------------------------------------------------------------


\section{Topic-based Annotations}
% https://onnx.ai/

The annotation of human-readable documents is a well-known problem in the Artificial Intelligence domain in general and Information Retrieval and Natural Language Processing fields in particular. There already exist a broad set of tools and frameworks able to analyze text for automatically producing such annotations, at very different levels of granularity: from minimal units such as terms and entities, to descriptors at the level of the entire collection such as  topics or summaries. For example, StanfordNLP~\cite{Manning2014TheToolkit} framework allows to perform different operations such as Part-of-Speech (PoS) tagging or Named Entity Recognition in various languages. Others like Mallet\footnote{\url{http://mallet.cs.umass.edu}} or SparkLDA\footnote{\url{https://spark.apache.org/mllib/}} perform topic modeling and clustering. We are focused on the transversal problem of making those standalone tools coexisting under the same solution. Being able to effectively integrating  them  under a common ecosystem helps to seamlessly obtain different kind of  annotations and boost the way those solutions can make sense of document collections.  
 
Certain systems among the research and industrial communities have already integrated some of the annotation tools introduced above. For example, \cite{gate2013} works with records from the biomedical domain, where robustness and high precision are prioritized. Therefore they rely on techniques supported by  GATE\footnote{\url{https://gate.ac.uk/}} framework, which widely supports hand-crafted, domain specific techniques such as rules or finite state transducers. On the other side of the spectrum we find \cite{chielang2012}, where the authors try to annotate text from a much noisier, sparser and error-prone medium: a tweet stream. Therefore they do not rely on any linguistic feature, due to the unpredictable way short social media post are written. We observe how each of those examples has very specific needs and leverages on certain annotation tools in order to accomplish the tasks it was originally created for. In both systems the involved components are highly coupled so they can not be easily extended to contemplate complementary annotation tools or alternative modules. 
%On the contrary, \textit{librAIry} advocates loosely interconnected components that make the architecture more reusable and expandable in other systems across domains.
 
One crucial problem regarding the re-usability and expansion possibilities of those systems and the tools they leverage on is the language they have been developed in. For example, Mallet uses Java, but others like spaCy \footnote{\url{https://spacy.io}} are python-based. To the best of our knowledge, there has not been any significant efforts on reconciling into a single architecture such heterogeneous set of tools, therefore minimizing the engineering effort and maximizing scalability of the system so it can be applied to very different domains and textual annotation tasks.

In addition, available annotation systems rely on certain storage solutions that are suited for some tasks but are less adequate others. For example \cite{furlong2008osirisv1} uses a relational database (MySQL\footnote{\url{https://www.mysql.com/}}) to ensure reliability and speed in managing the indexed information. In \cite{rizzo20153cixty},  the authors leverage on Virtuoso triple-store to provide native graph operations over the data. But new requirements may be considered for those systems so different storage needs can come into play.  For example, column oriented databases (Cassandra\footnote{\url{http://cassandra.apache.org}}) can help to better handle high-volume queries on specific data fields. Same goes with text oriented indexes such as ElasticSearch \footnote{\url{https://www.elastic.co}}, which can provide customized text-based search operations over the available information. 
%\textit{librAIry} straightforward supports the coexistence of different storage solutions, so it can be agnostic to the kind of underlying storage modules implemented. Thanks to the distributed nature of the proposed architecture,  different databases can be synchronized under the same common environment working together to store and deliver results in a more efficient manner.


\section{Document Embeddings}

%\subsection{Probabilistic Topic Models}

%\subsection{Multilingual Topic Models}


\section{Document Similarity}
..
%\subsection{Distance Metrics}

%\subsection{Hash Functions}

\section{Summary}
..
